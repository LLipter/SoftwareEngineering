\documentclass[11pt]{article}
\usepackage[english]{babel}
\usepackage[utf8]{inputenc}
\usepackage{fancyhdr}

\def\Name{Ran Liao}
\def\Topic{Module}

\title{\textbf{\Topic}}
\author{\Name}
\markboth{Notes on \Topic\ }{Notes on \Topic\ }
\date{\today}
 
\pagestyle{fancy}
\fancyhf{}
\rhead{\date{\today} }
\lhead{Notes on \Topic\ }
\rfoot{\thepage}

\textheight=9in
%\textwidth=6.5in
\topmargin=-.75in
%\oddsidemargin=0in
%\evensidemargin=0in
 
\begin{document}
\maketitle
\noindent\makebox[\linewidth]{\rule[8pt]{5in}{0.5pt}}

\section*{Definition}

A lexically contiguous sequence of program statements, bounded by boundary elements, with an aggregate identifier.

\section*{Cohesion - the higher the better}

Cohesion represents the degree of interaction within a module. 

\begin{enumerate}
	\item \textbf{Coincidental Cohesion}
	
	A module has coincidental cohesion if it performs multiple, completely unrelated actions.
	
	\item \textbf{Logical Cohesion}
	
	A module has logical cohesion when it performs a series of related actions, one of which is selected by the calling module.
	
	\item \textbf{Temporal Cohesion}
	
	A module has temporal cohesion when it performs a series of actions related in time.
	
	\item \textbf{Procedural Cohesion}
	
	A module has procedural cohesion if it performs a series of actions related by the procedure to be followed by the product.

	\item \textbf{Communicational Cohesion}
	
	A module has communicational cohesion if it performs a series of actions related by the procedure to be followed by the product, but in addition all the actions operate on the same data.
	
	\item \textbf{Functional Cohesion}
	
	A module with functional cohesion performs exactly one action.
	
	\item \textbf{Informational Cohesion}
	
	A module has informational cohesion if it performs a number of actions, each with its own entry point, with independent code for each action, all performed on the same data structure.
	
\end{enumerate}

\section*{Coupling - the lower the better}

Coupling represents the degree of interaction between two modules.

\begin{enumerate}
	\item \textbf{Content Coupling}
	
	Two modules are content coupled if one directly references contents of the other.

	\item \textbf{Common Coupling}
	
	Two modules are common coupled if they have write access to global data.
	
	\item \textbf{Control Coupling}
	
	Two modules are control coupled if one passes an element of control to the other.
	
	\item \textbf{Stamp Coupling}
	
	Two modules are stamp coupled if a data structure is passed as a parameter, but the called module operates on some but not all of the individual components of the data structure.

	\item \textbf{Data Coupling}
	
	Two modules are data coupled if all parameters are homogeneous data items (simple parameters, or data structures all of whose elements are used by called module).
	
\end{enumerate}

\section*{Information Hiding - the higher the better}

Ensure that implementation details are not visible outside the module in which they are declared.


\end{document}